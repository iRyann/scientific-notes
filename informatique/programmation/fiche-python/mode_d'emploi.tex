\documentclass[12pt,a4paper,fleqn]{article}

%%%%%%%%%%%%%%%%%%%%%% LES PACKAGES
\input{packages_MP2I.tex}
\usepackage{pgffor} 
%tikz juste pour le foreach pour présenter la palette de couleurs

%%%%%%%%%%%%%%%%%%%%%% MISE EN FORME
%%%%%%%%%%%%%%%%%%%%%%%%%%%%%%%%%%%%%%%%%%%%
% MISE EN FORME DU TITRE 
%%%%%%%%%%%%%%%%%%%%%%%%%%%%%%%%%%%%%%%%%%%%
\newcommand{\titreCopie}[2]{
\noindent
\textsc{Lettres Philosophie}
\hfill
#2
%\\%[0.4cm]
\begin{center}
\LARGE
\hrule
\vspace{.4cm}
#1 \\[0.4cm]
\hrule
\end{center}
\vspace*{0.4cm}
}

\newcommand{\titrecours}[1]{
\noindent
\begin{center}
\LARGE
\hrule
\vspace{.4cm}
#1 \\[0.4cm]
\hrule
\end{center}
\vspace*{0.4cm}
}

%%%%%%%%%%%%%%%%%%%%%%%%%%%%%%%%%%%%%%%%%%%%
% MISE EN FORME DES PIEDS DE PAGE
%%%%%%%%%%%%%%%%%%%%%%%%%%%%%%%%%%%%%%%%%%%%
\AtEndDocument{\label{lastpage}}
\lhead{}
\chead{}
\rhead{}
\lfoot{ 
Ryan Bouchou
}
\cfoot{}
\rfoot{\thepage/\pageref{lastpage}}
\renewcommand{\headrulewidth}{0pt}
\fancyhfoffset{0.7cm}



%%%%%%%%%%%%%%%%%%%%%%%%%%%%%%%%%%%%%%%%%%%%
% REMISE EN FORME DES SOUS-SOUS-SECTIONS  
%%%%%%%%%%%%%%%%%%%%%%%%%%%%%%%%%%%%%%%%%%%%

% section = exercices
%\renewcommand{\thesection}{\centering \arabic{section}}
% sous-section = question
%renewcommand{\thesubsection}{\arabic{subsection}}

\newcommand{\ssection}[1]{%
 
\begin{center}
    \hline
    
    \section*{\centering\normalfont\scshape #1}
    
    \hline
\end{center}
  }


\titleformat{\subsubsection}
   {\normalfont\fontsize{13pt}{13pt}\selectfont\bfseries}% apparence commune au titre et au numéro
   {\thesubsubsection}% apparence du numéro
   {1em}% espacement numéro/texte
   {}% apparence du titre

\titleformat{\subsubsubsection}
   {\normalfont\fontsize{11pt}{12pt}\selectfont\bfseries}% apparence commune au titre et au numéro
   {\thesubsubsubsection}% apparence du numéro
   {1em}% espacement numéro/texte
   {}% apparence du titre
%%%%%%%%%%%%%%%%%%%%%%%%%%%%%%%%%%%%%%%%%%%%
% MISE EN FORME DU CODE AVEC MINTED
%%%%%%%%%%%%%%%%%%%%%%%%%%%%%%%%%%%%%%%%%%%%

%pour le style des numéros de ligne minted
\renewcommand{\theFancyVerbLine}{\sffamily
\textcolor{expli}{\scriptsize
\oldstylenums{\arabic{FancyVerbLine}}}}

%pour éviter l'italique 
%car apparaissaient en italique à la fois les commentaires et les includes
\newtoggle{inminted}
\AtBeginEnvironment{minted}{\let\itshape\relax}

%paramètres pour le c
\setminted[c]{
	%--fond et cadre	
	bgcolor = black!3!white,
	frame = leftline,
	framesep = 6pt,
	rulecolor= expli,
	%--numéro de ligne
	linenos=true,
	numbersep=4pt,
	%stepnumber=2,
	xleftmargin=20pt,%pour que les chiffres débordent pas
	%--découpage des longues lignes
	breaklines=true,
	%--tabulations
	tabsize=2,
	}

%%
\setminted[csyntax]{
	%--fond et cadre	
	bgcolor = black!3!white,
	frame = leftline,
	framesep = 6pt,
	rulecolor= expli,
	%--numéro de ligne
	linenos=false,
	numbersep=4pt,
	%stepnumber=2,
	xleftmargin=20pt,%pour que les chiffres débordent pas
	%--découpage des longues lignes
	breaklines=true,
	%--tabulations
	tabsize=2,
	}

%paramètres pour le ocaml
\setminted[ocaml]{
	%--fond et cadre	
	%bgcolor = black!3!white,
	frame = leftline,
	framesep = 6pt,
	rulecolor= expli,
	%--numéro de ligne
	linenos=true,
	numbersep=4pt,
	%stepnumber=2,
	xleftmargin=20pt,%pour que les chiffres débordent pas
	%--découpage des longues lignes
	breaklines=true,
	%--tabulations
	tabsize=2,
	}

 %paramètres pour le ocaml
\setminted[bash]{
	%--fond et cadre	
	bgcolor = black!3!white,
	frame = leftline,
	framesep = 6pt,
	rulecolor= expli,
	%--numéro de ligne
	linenos=true,
	numbersep=4pt,
	%stepnumber=2,
	xleftmargin=20pt,%pour que les chiffres débordent pas
	%--découpage des longues lignes
	breaklines=true,
	%--tabulations
	tabsize=2,
	}







%on change un peu cette mise en forme ici
%car le mode d'emploi n'est pas une copie...
\renewcommand{\thesection}{\arabic{section}}
\renewcommand{\thesubsection}{\arabic{section}.\arabic{subsection}}
\lfoot{ 
Informatique - MP2I Lycée Fermat - 2021/2022
}

%%%%%%%%%%%%%%%%%%%%%%% LES COMMANDES 
\input{commandes_MP2I.tex}

%%%%%%%%%%%%%%%%%%%%%%% LA PALETTE DE COULEURS
\input{palette_aef.tex}

%%%%%%%%%%%%%%%%%%%%%%%%%%%%%%%%%%%%%%%
\begin{document}
\pagestyle{fancy} %active les pieds de pages

%%%%%%%%%%%%%%%%%%%%%%%%%%%%%%%%%%%%%%%
%%%%%%%%%%%%%%%%%%%%%%%%%%%%%%%%%%%%%%%
\section{Généralités}
%======================================
\subsection{Contenu du modèle}
%======================================
Cette archive a été crée à destination d'élèves de classe de MP2I
(du lycée Fermat) souhaitant rédiger leurs devoirs en latex.
Elle peut être partagée avec toute personne qui pourrait trouver ça utile.
Elle contient :
\begin{itemize}
\flch Ce mode d'emploi au format PDF : 
\texttt{mode\_d'emploi.pdf}
\flch Le code source de ce document, rédigé lui-même en latex : \texttt{mode\_d'emploi.tex}
\flch La liste des imports de package utiles :
\texttt{packages\_MP2I.tex}
\flch Du code latex pour paramétrer la mise en page, 
et la mise en forme :
\texttt{mise\_en\_forme\_MP2I.tex}
\flch Du code latex pour définir des commandes (des raccourcis) :
\texttt{commandes\_MP2I.tex}
\flch Une liste de définitions de couleurs :
\texttt{palette\_aef.tex}
\flch Un Makefile pour nettoyer automatiquement les fichiers auxiliaires
générés dans le dossier courant (se lance dans un terminal linux
avec \mintinline{bash}{make clean}) :
\texttt{Makefile}
\flch Un document latex (presque) prêt à être rempli :
\texttt{copie\_vide.tex}
\flch Le document PDF généré par ce document latex :
\texttt{copie\_vide.pdf}
\end{itemize}


%======================================
\subsection{Avant de commencer}
%--------------------------------------
Avant d'utiliser la copie \texttt{copie\_vide.tex}
il faut remplir votre nom et votre prénom à deux endroits dans
le fichier \texttt{mise\_en\_forme\_MP2I.tex}.
Pour trouver ces deux endroits dans le code,
vous pouvez rechercher \textsf{Ctrl+F} le mot \texttt{remplir}.\\
Il faut alors supprimer le  
\mintinline{latex}{\textcolor{red}{$\leftarrow$ à remplir}},
et évidemment remplacer \texttt{nom} et \texttt{prénom} par les vôtres.
Ces deux changements devraient 
s'observer sur la titre en haut de la première page
et sur le pied de page sur toutes les pages
(après compilation, bien sûr).

%======================================
\subsection{Organisation de la copie}
%--------------------------------------
La commande qui génère le haut de la première page est
\mintinline{latex}{\titreCopie}.
Elle est définie dans le fichier \texttt{mise\_en\_forme\_MP2I.tex},
vous pouvez donc la personnaliser.
En l'état, elle prend deux arguments,
le premier est le titre du rendu, 
le second est la date qui apparaît en haut à droite.
\\[0.2cm]
Chaque section du document correspond à un exercice,
et chaque sous-section à une question de cet exercice.
La numérotation est automatique,
à chaque commande \mintinline{latex}{\section{}}
le numéro d'exercice est incrémenté 
et le numéro de question remis à 0.
On peut toutefois sauter un exercice grâce à la commande
\mintinline{latex}{\addtocounter{section}{1}},
voire commencer à l'exercice $n+1$ grâce à la commande
\mintinline{latex}{\setcounter{section}{n}}.
De même, le numéro de question est incrémenté 
à chaque \mintinline{latex}{\subsection{}},
mais on peut sauter une question avec
\mintinline{latex}{\addtocounter{subsection}{1}},
voire commencer à la question $n+1$ avec
\mintinline{latex}{\setcounter{subsection}{n}}.

%======================================
\subsection{Divers}
%--------------------------------------
Les commentaires en latex sont précédés du symbole \texttt{\%}
(pour obtenir le symbole \texttt{\%} on tape donc
\mintinline{latex}{\%}).
Avec TexMaker on peut utiliser \textsf{Ctrl+T} pour commenter
et \textsf{Ctrl+U} pour dé-commenter.
\\[0.2cm]
À la fin d'une ligne non vide,
on réalise un saut de ligne simple avec \mintinline{latex}{\\},
ou un saut de ligne suivi d'un espace de 8mm (par exemple)
avec \mintinline{latex}{\\[8mm]}
(pour obtenir le symbole \texttt{\textbackslash} 
on tape \mintinline{latex}{\textbackslash} en mode texte,
ou  \mintinline{latex}{\backslash} en mode maths).
On réalise un saut de page avec
\mintinline{latex}{\newpage}.

%%%%%%%%%%%%%%%%%%%%%%%%%%%%%%%%%%%%%%%
\section{Inclure du code}
%======================================
Pour inclure du code source dans un document 
avec une coloration syntaxique adaptée au langage,
j'utilise le package \texttt{minted}
(à préférer à \texttt{listings} pour le Ocaml notamment).
\\
Une fois ce package importé, 
on peut afficher du code en utilisant les paramètres par défaut.
Cependant, j'ai redéfini certains paramètres pour 
le C et le OCaml (\ie les langages au programme de MP2I).
Ces définitions se trouvent dans le fichier 
\texttt{mise\_en\_forme\_MP2I.tex}
si vous voulez les modifier.\\

\noindent
Il y a trois façon d'afficher du code.
\begin{itemize}
\flch 
"En ligne", c'est-à-dire intégré au texte,
avec la commande
\mintinline{latex}{\mintinline{lang}{code}}.
\\
Cela donne par exemple en C \mintinline{c}{printf("hello");},
ou en OCaml \mintinline{ocaml}{let a:int = 12;}.
\flch
Dans un bloc, à partir du code écrit directement dans le code latex,
à l'intérieur de l'environnement \texttt{minted},
c'est à dire comme ceci :
\inputminted[firstline=1, lastline=3]{latex}{exemple_code.txt}
\flch
Dans un bloc, à partir du code d'un autre fichier,
entre les lignes a et b
avec la commande suivante :
\mintinline{latex}{\inputminted[firstline=a, lastline=b,firstnumber=1]{langage}{fichier}}.
\end{itemize}
On donne ci-dessous deux exemples de code dans un bloc,
le rendu étant le même que l'on utilise
\mintinline{latex}{\inputminted} ou 
\mintinline{latex}{\begin{minted}....}.\\

\begin{minipage}[t]{0.48\textwidth}
\inputminted[firstline=5, lastline=8,firstnumber=1]{c}{exemple_code.txt}
\end{minipage}
\begin{minipage}[t]{0.48\textwidth}
\inputminted[firstline=10, lastline=14,firstnumber=1]{ocaml}{exemple_code.txt}
\end{minipage}

%%%%%%%%%%%%%%%%%%%%%%%%%%%%%%%%%%%%%%%
\section{Faire un tableau}
%======================================
Pour un tableau on peut utiliser
l'environnement \texttt{tabular}.
On précise le format du tableau
en donnant le type de ses colonnes :
\texttt{c} pour centrée,
\texttt{r} pour alignée à droite,
\texttt{l} pour alignée à gauche.
La largeur de ces colonnes s'adapte à leur contenu,
au contraire une colonne de type \mintinline{latex}{p{xcm}}
est de largeur fixée : \texttt{x}cm.
On peut aussi redéfinir l'espace inter-colonne 
en ajoutant \mintinline{latex}{@{nouvel inter colonne}},
ou préciser que l'on souhaite un trait vertical (\resp deux)
entre deux colonnes
avec \mintinline{latex}{|} (\resp \mintinline{latex}{||}).
Enfin, pour obtenir une ligne horizontale 
on utilise \mintinline{latex}{\hline}.
On donne ci-dessous deux exemples (code à gauche, rendu à droite).
\\


\noindent
\begin{minipage}{0.54\textwidth}
\begin{minted}{latex}
\begin{tabular}{@{}rcl}
alors & bonjour & Ça va?\\
$a$ & $\mathbb{B}$ & CoucouHibouCoucouHibou\\
\end{tabular}
\end{minted}
\end{minipage}
\hspace*{0.1cm}\color{expli}\vrule \color{black}\hspace*{0.1cm}
\begin{minipage}{0.46\textwidth}
\begin{tabular}{@{}rcl}
alors & bonjour & Ça va?\\
$a$ & $\mathbb{B}$ & CoucouHibouCoucouHibou\\
\end{tabular}
\end{minipage}\\[0.6cm]

\noindent
\begin{minipage}{0.54\textwidth}
\begin{minted}{latex}
\begin{tabular}{p{4.6cm}|c||c@{ ! }c|}
titre long qui va à la ligne tout seul &&&\\
\hline \hline\\[-0.2cm]
$\sqrt{(1+2)*(3+4)} $ & 1 & 2 & 3\\
\hline
aaa & $aaa$ & \texttt{aaa} & A\\
\end{tabular}
\end{minted}
\end{minipage}
\hspace*{0.1cm}\color{expli}\vrule \color{black}\hspace*{0.1cm}
\begin{minipage}{0.46\textwidth}
\begin{tabular}{p{4.6cm}|c||c@{ ! }c|}
titre un peu longue qui va à la ligne tout seul &&&\\
\hline \hline
${(1+2)*(3+4)} $ & 1 & 2 & 3\\
\hline
aaa & $aaa$ & \texttt{aaa} & A\\
\end{tabular}
\end{minipage}

\vspace*{-0.2cm}

\newpage
%%%%%%%%%%%%%%%%%%%%%%%%%%%%%%%%%%%%%%%
\section{Expressions mathématiques}
%======================================
On donne ci-dessous des expressions mathématiques 
que vous pourriez être amenés à écrire.
Pour voir comment elles sont codées,
regardez le code source dans le fichier \texttt{mode\_d'emploi.tex}.
Si vous utilisez TexMaker et  son afficheur de PDF interne,
vous pouvez facilement retrouver la portion de code correspondante
en faisant \textsf{Control+Clic} sur le document 
(il faut cependant avoir compilé le code source).

%======================================
\subsection{Ensemble}
$$\left\lbrace 
\sum_{i=1}^n x_i + 3 \sqrt{25+3}
\enskip \middle| \enskip
(x_i)_{i\in[1..n] } \in A^n
\right\rbrace$$

%======================================
\subsection{Fonction}
$$f= \left( \begin{array}{lll}
A & \rightarrow & B\\
x & \mapsto & x+2\\
\end{array}\right)$$

%======================================
\subsection{Disjonction de cas}
$
f(x) = \begin{cases}
a & \text{si } x=0\\
c & \text{sinon}\\
\end{cases}
$
\hspace*{0.6cm}
ou
\hspace*{0.6cm} 
$f(x) = \left\lbrace \begin{tabular}{ll}
$a$ & si $b=0$\\
$c$ & sinon\\
\end{tabular}\right.
$
\hspace*{0.6cm}
ou
\hspace*{0.6cm} 
$f(x) = \left\lbrace \begin{array}{ll}
a & \text{si } x=0\\
c & \text{sinon}\\
\end{array}\right.
$

%======================================
\subsection{Règles de construction}
Pour les arbres 2-3-4 par exemple 
$V|^0_{\{_\}}$, $N^2|^2_S$,
$N^3|^3_{S^\leqslant}$ et
$N^4|^4_{S^{\leqslant\,\leqslant}}$


%======================================
\subsection{Accolade en dessous}
$$ \underbrace{a+b+c}_{=125} + d = 125 + d$$

%======================================
\subsection{Barrer des termes}
$$ 125 + a + b -125 =  \cancel{125} + a + b - \cancel{125} = a + b $$

%%%%%%%%%%%%%%%%%%%%%%%%%%%%%%%%%%%%%%%
\section{Aligner  des équations}
%=====================================
\textcolor{red}{à venir}
%%%%%%%%%%%%%%%%%%%%%%%%%%%%%%%%%%%%%%%
\section{Catalogue de symboles}
%=====================================
\textcolor{red}{à venir}

\newpage
%%%%%%%%%%%%%%%%%%%%%%%%%%%%%%%%%%%%%%%
\section{Les couleurs}
%======================================
\subsection{Astuce}
\noindent
Saviez-vous qu'on peut facilement obtenir des nouvelles 
couleurs en mixant celles existantes? 
Il suffit de faire \texttt{c1!50!c2} pour mélanger c1 et c2.
Exemple:
\textcolor{blue!15!green}{blue!15!green}
\textcolor{blue!50!green}{blue!50!green}
\textcolor{blue!80!green}{blue!80!green}
%======================================
\subsection{Ma palette de couleurs}
\noindent
\noindent
Les couleurs ci-dessous sont définies dans le fichier \texttt{palette\_aef.tex}.
\\
\foreach \coul in {jaune,oranger,orange}{
\noindent
\textcolor{\coul}{un aperçu de la couleur \textbf{\coul}  \enskip \hrulefill}\\
} 


\foreach \coul in {vert,herbe,vertFonce}{
\noindent
\textcolor{\coul}{un aperçu de la couleur \textbf{\coul}  \enskip \hrulefill}\\
} 

\foreach \coul in {vertdEau,turquoise,monCyan,turquoiseFonce}{
\noindent
\textcolor{\coul}{un aperçu de la couleur \textbf{\coul}  \enskip \hrulefill}\\
} 

\foreach \coul in {marine, doubleu,lavande,mauve,lilas}{
\noindent
\textcolor{\coul}{un aperçu de la couleur \textbf{\coul}  \enskip \hrulefill}\\
} 

\foreach \coul in {violet, violette,prune,prunelle}{
\noindent
\textcolor{\coul}{un aperçu de la couleur \textbf{\coul}  \enskip \hrulefill}\\
} 

\foreach \coul in {roseAcidule,magenta,rose,framboise,grenat}{
\noindent
\textcolor{\coul}{un aperçu de la couleur \textbf{\coul}  \enskip \hrulefill}\\
} 

\foreach \coul in {brique, briqueRouge}{
\noindent
\textcolor{\coul}{un aperçu de la couleur \textbf{\coul}  \enskip \hrulefill}\\
} 

\foreach \coul in {saumon}{
\noindent
\textcolor{\coul}{un aperçu de la couleur \textbf{\coul}  \enskip \hrulefill}\\
} 

\foreach \coul in {ocre,chamois,taupe}{
\noindent
\textcolor{\coul}{un aperçu de la couleur \textbf{\coul}  \enskip \hrulefill}\\
} 

%%%%%%%%%%%%%%%%%%%%%%%%%%%%%%%%%%%%%%%%
%\section{Définir des nouvelles commandes}
%\textcolor{red}{à venir}


\end{document}
