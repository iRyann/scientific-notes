\section{Fondements}
Les mathématiques sont une discipline qui exige à la fois clarté, rigueur et originalité.
Ainsi, il convient de maîtriser un certain nombre de conventions afin de rendre son propos cohérent
et intelligible. Dans ce contexte, on se propose de présenter les notations usuelles nécessaires
à notre pratique des mathématiques.
\subsection{Les ensembles de nombres}
Les ensembles sont traités spécifiquement dans le chapitre \cref{ensembles}, pour autant, on procède
à un certain nombre de rappels ici.
\subsubsection{Ensembles usuels}
\begin{itemize}
    \item On appelle ensemble des \important{entiers naturels}, nôté $\mathbb{N}$, l'ensemble composé
    des nombres entiers positifs ou nuls.
    \item On appelle ensemble des \important{entiers relatifs}, nôté $\mathbb{Z}$, l'ensemble composé
    des entiers de signe quelconque. 
    \item On appelle ensemble des \important{rationnels}, nôté $\mathbb{Q}$, l'ensemble composé
    des nombres pouvant s'écrire sous la forme d'une fraction.
    \item On appelle ensemble des \important{décimaux}, nôté $\mathbb{D}$, l'ensemble composé
    des nombres à virgule.
    \item On appelle ensemble des \important{réels}, nôté $\mathbb{R}$, l'ensemble composé
    des nombres réels (\textit{i.e. quelconques})
    \item On appelle ensemble des \important{complexes}, nôté $\mathbb{C}$, l'ensemble composé
    des nombres complexes. (\textit{hors programme})
\end{itemize}
\begin{coder}
    De plus, on a l'inclusion suivante entre les ensembles précédemment cités:
$$\mathbb{N} \subset \mathbb{Z} \subset \mathbb{D} \subset \mathbb{Q} \subset \mathbb{R} \subset \mathbb{C}$$
\end{coder}
Par ailleurs, étant donné un ensemble $\mathbb{E}$ parmi les ensembles usuels, on introduit
les notations:
\begin{itemize}
    \item $\mathbb{E}^* = \mathbb{E}\setminus\{0\}$
    \item $\mathbb{E}^+$ l'ensemble des termes positifs ou nuls de $\mathbb{E}$.
    \item $\mathbb{E}^-$ l'ensemble des termes négatifs de $\mathbb{E}$.
\end{itemize}
Ce faisant, on a par exemple $\mathbb{R}^{*}_{+}$ l'ensemble des réels strictement positifs, $\mathbb{Z}^{*}$
l'ensemble des entiers relatifs privé de 0, et ainsi de suite. 
\subsubsection{Intervalles}
Considérons $a,b\in\mathbb{R}$ tels que $a \leq b$. Dès lors, on note:
\begin{itemize}
    \item $[a,b]$ l'ensemble des nombres réels compris entre $a$ et $b$ inclus. Autrement-dit
    l'ensemble des réels $x$ tels que $a \leq x \leq b$. 
    \item $]a,b]$ l'ensemble des nombres réels compris entre $a$ exclu et $b$ inclus. Autrement-dit
    l'ensemble des réels $x$ tels que $a < x \leq b$. De façon similaire, on définit $[a,b[ et ]a,b[$.
\end{itemize}
De plus, on définit les intervalles d'entiers. Considérons $a,b\in\mathbb{N}$ tels que $a \leq b$. 
Dès lors, on note:
\begin{itemize}
    \item $[\![a,b]\!]$ l'ensemble des nombres entiers compris entre $a$ et $b$ inclus. Autrement-dit
    l'ensemble des entiers $x$ tels que $a \leq x \leq b$. 
    \item $]\!]3;4]\!]$ l'ensemble des nombres entiers compris entre $a$ exclu et $b$ inclus. Autrement-dit
    l'ensemble des entiers $x$ tels que $a < x \leq b$. De façon similaire, on définit $[\![a,b[\![$ et $]\!]a,b[\![$.
\end{itemize}
On note $+\infty$ et $-\infty$ les infinis, $\infty$ par abus lorsque le signe est obvie, ainsi que
$\pm\infty$ lorsque l'on souhaite dénoter les deux cas. Par ailleurs, \important{lorsqu'un infinis constitue l'une des bornes
d'un intervalle, la borne qui lui est adjointe est nécessairement ouverte}. Par exemple, on note 
$[0,+\infty[=\mathbb{R}^+$.\\
\begin{codeb}
    \textbf{Remarque}
    \begin{itemize}
        \item Si $a,b\in\mathbb{R}$ et $a \leq b$ ; alors, $[b,a]=\emptyset$. Similairement pour les intervalles d'entiers.
        \item Si $a,b\in\mathbb{R}$ et $a = b$ ; alors, $[a,b]=\{a\}=\{b\}$. Similairement pour les intervalles d'entiers.
        \item Si $a,b\in\mathbb{N}$ et $a \leq b$ ; alors, $[\![a,b]\!]=\{a, a+1, \dots, b-1, b\}=\{x\in\mathbb{N} : a \leq x \leq b\}$.
    \end{itemize}
\end{codeb}
\subsection{Manipulation des symboles}
\textit{Pour la suite, on suppose fixé deux ensembles $I$ et $E$ tous deux non vides.}
\subsection{Famille}
Une famille est une fonction d'un ensemble $I$ dans un ensemble $E$ permettant
d'indexer les éléments de $E$ à partir de ceux de $I$. On la note $(x_i)_{i\in I}$,
ou $(x_i)$ lorsque l'indexation est obvie. De plus, $x_i$ est le $i^{eme}$ termes de la famille. 
\subsubsection{Sommation}
Lorsque l'on travaille sur des ensembles, nous pouvons être emmener à sommer leurs éléments. 
Si on considère une famille $(x_i)$ d'éléments de $E$, alors:
$$\sum_{i\in I}x_i \text{ est la somme des éléments } x_i, \text{ pour tout } i \text{ dans } I$$
\begin{codeb}
    \textbf{Remarque}\\
    \begin{itemize}
        \item Il est important de noter que le choix de i importe peu ; subséquemment, on a:
        $$\sum_{i\in I}x_i = \sum_{j\in I}x_j = \sum_{\beta \in I}x_{\beta}$$
        On dit que i est une \emph{variable muette}. En revanche, il est important de discerner
        ce qui est fixé et ce qui ne l'est pas. Conséquentiellement, $\sum_{i\in[1..i]}x_i$ n'a pas de sens.
        \item Lorsque $E$ est fini de cardinal $n$, on considère $I=\{1,\dots,n\}$. Dans le cas contraire, généralement $I=\mathbb{N}$. 
    \end{itemize}
    
\end{codeb}
\begin{exemple} Un somme usuelle pour illustrer le principe:
$$\sum_{k=1}^{n}k=1+2+\dots+n=\sum_{k\in[\![1,n]\!]}k=\frac{n(n+1)}{2}$$
\end{exemple}
\subsection{Logique}
La logique est une branche fondamentale des mathématiques qui sous-tend notre capacité à étayer nos raisonnements en formalisant la notion de démonstration. À ce titre, on dispose de plusieurs outils de rédaction afin de mettre en oeuvre nos démonstrations.
\begin{definition}
    On appelle \emph{proposition} un énoncé mathématique possédant une valeur de vérité, vrai ou faux.
\end{definition}
\begin{exemple}
    "\textit{Dans tout triangle rectangle, le carré de l’hypoténuse est égale à la somme
des carrés des deux autres côtés}" est une proposition vraie.
\end{exemple}
\begin{definition}
    On appelle \emph{axiome} une proposition admise comme vraie.
\end{definition}
\subsubsection{Opérateurs}
\subsubsubsection{Non, ou, et}
Les opérateurs logiques permettent de former des propositions par \emph{induction} en combinant d'autres propositions. Considérons deux propositions $P$ et $Q$.
\begin{itemize}
    \item \important{non $P$} : est la proposition constituée de la négation de $P$
    
    \begin{table}[h]
        \centering
        \begin{tabular}{@{ }c | c }
        P & $\lnot P$\\
        \hline 
        T & \textcolor{red}{F} \\
        F & \textcolor{red}{T} \\
        \end{tabular}
        \caption{Table de vérité de la négation}
        \label{tab:negation}
    \end{table}
    
    \item \important{$P$ ou $Q$} : est la proposition qui est vrai si au moins l'une de ses deux composantes l'est.
    
    \begin{table}[h]
        \centering
        \begin{tabular}{@{ }c@{ }@{ }c | c@{ }@{ }c@{ }@{ }c@{ }@{ }c@{ }@{ }c}
        P & Q &  & P & $\lor$ & Q & \\
        \hline 
        T & T &  &  & \textcolor{red}{T} &  & \\
        T & F &  &  & \textcolor{red}{T} &  & \\
        F & T &  &  & \textcolor{red}{T} &  & \\
        F & F &  &  & \textcolor{red}{F} &  & \\
        \end{tabular}
        \caption{Table de vérité de la disjonction}
        \label{tab:disjonction}
    \end{table}
    
    \item \important{$P$ et $Q$} : est la proposition qui est vrai si $P$ et $Q$ le sont toutes deux.
    \begin{table}[h]
        \centering
        \begin{tabular}{@{ }c@{ }@{ }c | c@{ }@{ }c@{ }@{ }c@{ }@{ }c@{ }@{ }c}
        P & Q &  & P & $\land$ & Q & \\
        \hline 
        T & T &  &  & \textcolor{red}{T} &  & \\
        T & F &  &  & \textcolor{red}{F} &  & \\
        F & T &  &  & \textcolor{red}{F} &  & \\
        F & F &  &  & \textcolor{red}{F} &  & \\
        \end{tabular}
        \caption{Table de vérité de la conjonction}
        \label{tab:conjontion}
    \end{table}
\end{itemize}
\newpage
\subsubsubsection{Implication et équivalence}
\begin{itemize}
    \item \important{$P\implies Q$} : est la proposition qui signifie que "si $P$ est vraie, alors $Q$ aussi" c'est à dire "P est fausse ou Q est vraie". On dit que \emph{$P$ implique $Q$}
    \begin{table}[h]
        \centering
        \begin{tabular}{@{ }c@{ }@{ }c | c@{ }@{ }c@{ }@{ }c@{ }@{ }c@{ }@{ }c}
        P & Q &  & P & $\rightarrow$ & Q & \\
        \hline 
        T & T &  &  & \textcolor{red}{T} &  & \\
        T & F &  &  & \textcolor{red}{F} &  & \\
        F & T &  &  & \textcolor{red}{T} &  & \\
        F & F &  &  & \textcolor{red}{T} &  & \\
        \end{tabular}
        \caption{Table de vérité de l'implication}
        \label{tab:implication}
    \end{table}
    \item \important{$P \Leftrightarrow Q$} : est la proposition qui signifie que $\big( $($P$ implique Q) et ($Q$ implique $P$)$\big)$. On dit que \emph{$P$ est équivalent $Q$}
    \begin{table}[h]
        \centering
        \begin{tabular}{@{ }c@{ }@{ }c | c@{ }@{ }c@{ }@{ }c@{ }@{ }c@{ }@{ }c}
        P & Q &  & P & $\Leftrightarrow$ & Q & \\
        \hline 
        T & T &  &  & \textcolor{red}{T} &  & \\
        T & F &  &  & \textcolor{red}{F} &  & \\
        F & T &  &  & \textcolor{red}{F} & T & \\
        F & F &  &  & \textcolor{red}{T} &  & \\
        \end{tabular}
        \caption{Table de vérité de l'équivalence}
        \label{tab:equiv}
    \end{table}
\end{itemize}
\begin{coder}
    L'implication et l'équivalence sont des \important{relations transitives}. Ainsi, si :
    \begin{itemize}
        \item ($P \implies Q$) et ($Q \implies R$) alors ($P \implies R$)
        \item  ($P \Leftrightarrow Q$) et ($Q \Leftrightarrow R$) alors ($P \Leftrightarrow R$)
    \end{itemize}
\end{coder}
\subsubsection{Réciproque et contraposée}
\begin{definition}
    La \emph{réciproque} de "$P$ implique $Q$" est "$Q$ implique $P$".
\end{definition}
\begin{definition}
    La \emph{contraposée} de "$P$ implique $Q$" est "non $Q$ implique non $P$".
\end{definition}
\begin{coder}
    \important{Une implication et sa contraposée sont équivalentes}. De fait, lorsque l'on souhaite montrer "$P$ implique $Q$", il peut être préférable de montrer "non $Q$ implique non $P$". On parle de \emph{raisonnement par contraposée}
\end{coder}
\subsubsection{Quantificateurs}
Lorsque l'on souhaite utiliser des variables dans nos raisonnements, il est nécessaire d'introduire celles-ci. Pour ce faire, on dispose de \emph{quantificateurs}. Considérons une propriété $P$ dépendant de $x\in \mathcal{D}$, notée $P(x)$.
\begin{itemize}
    \item Lorsque $P$ est vraie pour toute valeur de $x$, on note : $\forall x\in \mathcal{D}, P(x).$ Ici, la virgule succédant le quantificateur signifie "on a".
    \item Lorsque $P$ est vraie pour au moins une valeur de $x$, on note : $\exists \,x \in \mathcal{D}, P(x).$ Ici, la virgule succédant le quantificateur signifie "tel que".
\end{itemize}