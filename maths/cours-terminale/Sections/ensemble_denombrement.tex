\section{Les ensembles}
\label{ensembles}
\subsection{Éléments généraux}
\begin{df}{Ensemble}
    Un ensemble est une collection d'objets appelés \textbf{éléments}, qui peuvent être en nombre fini ou non.
\end{df}
Un ensemble peut se définir de deux manières :
\begin{itemize}
    \item En donnant la liste explicite et exhaustive de ses éléments. (Raisonnablement dans le cas des ensembles finis)\\
    Exemple: $E=\{6,8,D,\%\}$
    \item Par compréhension : lorsque les éléments vérifient une propriété particulière.\\
    Exemple: $ E=\{x\in\mathbb{R} : x^{2}+x+1=0\} ~ i.e ~ Rac(X^{2}+X+1)$
\end{itemize}
\noindent
On note $\emptyset$ l'ensemble vide, ne contenant donc aucun élément. Par ailleurs, un certain nombre d'ensembles de références sont nécessaires ; à savoir: $(\mathbb{N},\mathbb{Z},\mathbb{Q},\mathbb{R},\mathbb{C})$
\noindent
On définit pour la suite E et F des ensembles quelconques, ainsi que $n\in\mathbb{N}^{*}$
\subsection{Opérations ensemblistes}
\begin{itemize}
    \item Appartenance: $x\in E$ si x appartient à E
    \item Inclusion: $E\subset F$ si E est inclus dans F ; i.e, E est un sous-ensemble de F
    \item Réunion: $E\cup F$ est l'ensemble des éléments appartenant à E ou F
    \item Intersection: $E\cap F$ est l'ensemble des éléments appartenant à E et à F
    \item Exclusion: $E\backslash F$ est l'ensemble des éléments de E qui n'appartiennent pas à F
    \item Différence symétrique: $E\triangle F$ est l'ensemble des éléments qui sont uniquement dans E et uniquement dans F. Autrement dit, $E\triangle F=(E\cup F)\backslash (E\cap F)$
\end{itemize}
Par ailleurs, on note P(E) \textbf{les parties de E}, l'ensemble des sous-ensembles de E.\\[0.3cm]
\warning ~ Si $B=\{1,7,8\}$ Attention à la différence entre $\{1\}\subset B ~et~ \{1\}\in P(B)$
\subsection{Cardinalité}
\begin{df}{Cardinal}
    On note Card(E), |E| ou encore \#E, le nombre d'éléments de E. On l'appelle \textbf{cardinal} de E.
\end{df}
\begin{df}{Ensembles deux à deux disjoints}
    Si $E_1,...,E_n$ sont deux à deux disjoints, alors $\forall i,j\in[1..n] ~ et ~i\neq j, Card((E_i\cap E_j))=0$
\end{df}
\begin{p}
    Si $E_1,...,E_n$ sont deux à deux disjoints et finis, alors $Card(E_1\cup ... \cup E_n)=\sum_{i=1}^{n}Card(E_i)$
\end{p}
\newpage
\subsection{Produit cartésien}
On appelle produit cartésien de n ensemble 
 $E_1,...,E_n$, l'ensemble \begin{center}
     $E_1\times...\times E_n=\{(x_1,...,x_n) ~|~ x_1\in E_1,...,x_n\in E_n\}$
 \end{center}
 dont les éléments sont des n\_uplets. On parle alors de couple, triplet, quadruplets etc...\\
 Si l'un des $E_i$ est vide alors, le produit cartésien l'est aussi. Enfin, si $E_1=...=E_n=E$ alors on note leur produit cartésien $E^{n}$

\begin{p}
    Soient $E_1,...,E_n$ des ensembles finis.\\
    $Card(E_1\times...\times E_n)=\prod\limits_{i=1}^n Card(E_{i})$
\end{p}
\subsection*{Exercices}
\subsubsection*{Exercices 1}
On considère le diagramme de Venn suivant, avec A,B,C trois parties d'un ensemble E;~et a,b,c,d,e,f,g,h des éléments de E.\\[.5cm]
\begin{minipage}{0.5\textwidth}
    \includegraphics[scale=2]{venn.png}
\end{minipage}
\begin{minipage}{0.5\textwidth}
    Dire si les affirmations suivantes sont vraies ou fausses:
    \begin{itemize}
        \item $g\in A\cap \bar B$
        \item $g\in\bar A\cap \bar B$
        \item $g\in\bar A\cup\bar B$
        \item $f\in C\backslash A$
        \item $e\in \bar A\cap\bar B\cap \bar C$
        \item $\{h,b\}\subset \bar A\cap\bar B$
    \end{itemize}
\end{minipage}
\subsubsection*{Exercice 2}
Soient 
A
,
B
,
C
 trois ensembles tels que 
A
$\cup$
B
=
B
$\cap$
C
. Montrer que 
A
$\subset$
B
$\subset$
C
.


\subsubsection*{Exercice 3}
Soient 
A
, 
B
 et 
C
 trois parties d'un ensemble 
E
. Pour 
X
$\subset$
E
, on note 
$X^c$
 le complémentaire de 
X
 dans 
E
. Démontrer les lois de Morgan suivantes :\\
\begin{center}
    \begin{array}{lll}
\mathbf{1.}\ (A\cap B)\cup C=(A\cup C)\cap (B\cup C)&&\mathbf{2.}\ (A^c)^c=A\\
\mathbf{3.}\ (A\cap B)^c=A^c\cup B^c&&\mathbf{4.}\ (A\cup B)^c=A^c\cap B^c.\\
\end{array}
\end{center}

\subsubsection*{Exercice 4}
Écrire l'ensemble des parties de E={a,b,c,d}.
\subsubsection*{Exercice 5}
On considère l'ensemble $D=\{(x,y)\in\mathbb R^2;\ x^2+y^2\leq 1\}$. Montrer qu'il ne peut pas s'écrire comme un produit cartésien de deux parties de $\mathbb{R}$. 
\subsubsection*{Exercice 6}
On considère $\sum_1 =\{0,1\}$ et  $\sum_2 =\{a,b,c,d,e\}$. On souhaite composer des mots de passes composé d'un chiffre et de 8 lettres. Quel est le nombre de mdp possibles ?\\ Quel est le nombres de mots de passes si on s'autorise à avoir un ou deux chiffres ?
\section{Dénombrement}
On considère dans cette partie un ensemble E de cardinal fini n et $0\leq k\leq n$:
\subsection{k\_Arrangements}

\begin{df}{Arrangement}
    On appelle k\_liste ou k\_Arrangements un k\_ uplet d'éléments de E tous différents. 
\end{df}
On assimile un k\_Arrangement au nombre d'issues lors d'un tirage sans remise de k éléments dans un ensemble à n éléments.
\begin{p}
    Le nombre de k\_Arrangements vaut $n(n-1)\dots (n-k+1)=\frac{n!}{(n-k)!}$
\end{p}
\begin{p}
    Le nombre de permutations de E vaut $n!$
\end{p}

\subsection{Combinaisons}
\begin{df}{k\_Combinaison}
    Partie de E à k éléments. 
\end{df}
On assimile le nombre de k-combinaison au nombre d'issues d'un tirage avec remise de k éléments dans un ensemble de cardinal n.
\begin{p}
    Le nombre de k-combinaisons de E est $\binom nk=\frac{n!}{k!(n-k)!}$\\
    Symétrie des coefficients binomiaux: $\binom nk=\binom n{n-k}$
\end{p}
\subsubsection*{Exercice 7}
On considère une course de karting comprenant n pilotes. Déterminer le nombre de podium possibles (on classera les 3 premiers).
\subsubsection*{Exercice 8}
On considère une course de karting comprenant n pilotes. Malheureusement, tous ne peuvent pas s'élancer sur la grille de départ en même temps. Déterminer le nombre de possibilités de sélectionner les k premiers qui partiront en premiers.  
\subsection{Triangle de Pascal}


\begin{center}
    
\begin{tikzpicture}[my rule/.style={line width=\myrulewidth},
                    my outline color/.initial=blue!75!black,
                    my text/.style={
                      color/.expanded={\pgfkeysvalueof{/tikz/my outline color}}},
                    my arrow/.style={
                      ->, line width=0.6pt,
                      draw/.expanded={\pgfkeysvalueof{/tikz/my outline color}}},
                    my highlight/.style={
                      draw/.expanded={\pgfkeysvalueof{/tikz/my outline color}},
                      fill=blue!15, line width=0.6pt, rounded corners=2pt}]

\matrix (mat) [
  my rule, draw, inner sep=0, matrix of math nodes,
  nodes={minimum width=\myCellSize,
         text height=0.6\myCellSize, text depth=0.4\myCellSize}]
  {
    \raisebox{-0.8ex}{$n$}\kern 0.3em\raisebox{0.7ex}{$k$}
           & 0      & 1 & 2 & 3 & 4 & 5   &[-2pt] \cdots \\
    0      & 1                                           \\
    1      & 1      & 1                                  \\
    2      & 1      & 2 & 1                              \\
    3      & 1      & 3 & 3  & 1                         \\
    4      & 1      & 4 & 6  & 4  & 1                    \\
    5      & 1      & 5 & 10 & 10 & 5 & 1                \\[-2pt]
    \vdots & \vdots &   &    &    &   &   &       \ddots \\
  };

\begin{scope}[my rule]
\draw (mat-1-1.north east) -- (mat-8-1.south east);
\draw (mat-1-1.south west) -- (mat-1-8.south east);
\draw ([shift={(0.5\myrulewidth, -0.5\myrulewidth)}]mat-1-1.north west) --
      (mat-1-1.south east);
\end{scope}

\path[my text] node [right=1cm of mat, label=above:{Formule de Pascal}]
  (formula)
  {%
    $\begin{aligned}
      \binom{n}{k} &\mathrel{+} \binom{n}{k+1} \\
                   &=           \binom{n+1}{k+1}
    \end{aligned}$%
  };

\begin{scope}[on background layer]
  \path[name path=p, my highlight]
    (mat-5-3.north west) -| coordinate (A)
    (mat-6-4.south east) -|
    (mat-5-3.south east) --
    (mat-5-3.south west)
    -- cycle;

  \pgfmathsetlengthmacro{\myWidth}{width("$\displaystyle \binom{n}{k}$")}
  \pgfmathsetlengthmacro{\myTotalHeight}{
    height("$\displaystyle \binom{n}{k}$") +
    depth("$\displaystyle \binom{n}{k}$")}
  \pgfmathsetlengthmacro{\myXshift}{\pgfkeysvalueof{/pgf/inner xsep} + \myWidth}
  \pgfmathsetlengthmacro{\myYshift}{\myTotalHeight +
                                    2*\pgfkeysvalueof{/pgf/inner ysep}}

  \path[my highlight]
    (formula.north west) -|
    (formula.south east) -|
    ([shift={(\myXshift,-\myYshift)}]formula.north west) coordinate (B) --
    (formula.north west |- B) --
    cycle;
  % Pour que la flèche parte vraiment de la bordure arrondie de p, il faut
  % ruser un peu. D'abord, il nous faut le point de départ :
  \path[name path=s] ([shift={(-10pt,-10pt)}]A) -- (A);
  \path[name intersections={of=p and s}];
\end{scope}

% Ensuite, on peut tracer la flèche.
\path[my arrow] (intersection-1) to[bend left=10]
                ($(formula.north west)!0.5!(formula.west)$);
\end{tikzpicture}
\end{center}
\subsubsection*{Démonstrations}
\begin{itemize}
    \flch\textbf{Méthode combinatoire}\\[5cm]
    \flch\textbf{Méthode algébrique}\\[5cm]
\end{itemize}




\newpage 
\begin{p}
    Pour n un entier natural, on a :$\sum_{k=0}^n=\binom{n}{k}=2^n$
\end{p}
\subsubsection*{Démonstrations}
\begin{itemize}
    \flch\textbf{Méthode combinatoire}\\[5cm]
    \flch\textbf{Méthode algébrique}\\[5cm]
\end{itemize}
\begin{p}
    Le nombre de parties d'en ensemble à n éléments vaut $2^n$
\end{p}
\subsubsection*{Exercice - Spécialité NSI}
\textit{On pourra aborder ici quelques notions sur les Langages}\\[0.3cm]
On considère $\sum_1 =\{0,1\}$ et  $\sum_2 =\{a,b,c,d,e\}$. On souhaite composer des mots de passes à partir du langage engendré par $L=\sum_{1}^{+} .\sum_{2}^{+}$ ; dont on restreindra leur taille à un entier n>1. On notera L' le langage subséquent. 
\begin{itemize}
    \item \textbf{Cas où n=2}\\
     Préciser la partie de L, notée L', qui nous intéresse ici à l'aide d'une description ensembliste par compréhension.\\
     Même question avec un produit cartésien. Donner alors le cardinal de cet ensemble. 
     \item \textbf{Cas où n>1}\\
     On considère u un mot de L'. Préciser sa décomposition et les caractéristiques de celles-ci.\\
     Donner le cardinal de L' en fonction de n, et des autres données de l'exercice. 
\end{itemize}

