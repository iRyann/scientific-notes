\ssection{Ensemble et dénombrement}
\label{tdEns}
\tocless\section{Formalisme des ensembles}
\tocless\subsection{Cours} 
\begin{minipage}{0.5\textwidth}
    \begin{itemize}
        \item $\pi\ldots\mathbb{Q}$
        \item $8.5\ldots\mathbb{R}$
        \item $\{6\}\ldots\mathbb{Z}$
        \item $(7,-9,5.8)\ldots~\ldots\ldots\ldots$
    \end{itemize}
\end{minipage}
\begin{minipage}{0.5\textwidth}
    \begin{itemize}
        \item $\mathbb{Q}\ldots\mathbb{R}$
        \item $\mathbb{N}\ldots\mathbb{Z}$
        \item $\mathbb{R}\ldots\mathbb{Z}$
        \item $\mathbb{R}^*\ldots\mathbb{R}$
    \end{itemize}
\end{minipage}
\tocless\subsection{Cours}Rappeler la définition d'un ensemble formulé par compréhension
\tocless\subsection{}
\begin{itemize}
    \item Donner l'ensemble des entiers naturels pairs
    \item Donner l'ensemble des entiers relatifs impairs
    \item Donner l'ensemble des entiers relatifs dont le reste de leur division par 3 vaut 2
    \item Soit $(n,S)\in\mathbb{N}^2$. Donner l'ensemble des n\_uplets dont la somme de leurs éléments vaut S.
\end{itemize}
\tocless\subsection{}
Soient $A=\{1,2,3\}$ et $B=\{0,1,2,4\}$. Donner les ensembles $A\cap B, A\cup B$ et $A\times B$
\tocless\subsection{}
Déterminer deux 3\_uplets de {0,1}. Combien en existe-t-il au total?
\tocless\subsection{}
Soient $A=[0,2]$ et $B=]1.5,3]$. Donner les ensembles $A\cap B, A\cup B$
\tocless\subsection{}
Soient A, B et C trois parties d'un ensemble E. Pour X$\subset$E, on note $X^c$ le complémentaire de X dans E. Démontrer les lois de Morgan suivantes :\\
\begin{center}
    \begin{math}
    \begin{array}{lll}
        \mathbf{1.}\ (A\cap B)\cup C=(A\cup C)\cap (B\cup C)&&\mathbf{2.}\ (A^c)^c=A\\
        \mathbf{3.}\ (A\cap B)^c=A^c\cup B^c&&\mathbf{4.}\ (A\cup B)^c=A^c\cap B^c.\\
    \end{array}
    \end{math}
\end{center}

\newpage
\tocless\section{Propriétés sur les ensembles}
\tocless\subsection{}
Soient $A=\{1,2,3\}$ et $B=\{0,1,2,4\}$. Donner $\#A, \#B, \#A\cup B$
\tocless\subsection{}
On réalise un sondage sur un ensemble d'individus afin de connaître les langages qu'ils maîtrisent. On reporte les résultats sur le diagramme de Venn suivant:\\[0.5cm]
\begin{minipage}{0.5\textwidth}
    \includegraphics[]{Imgs/images.jpg}
\end{minipage}
\begin{minipage}{0.5\textwidth}
Pour chacune des questions, on donnera une notation ensembliste afin de traduire l'énoncer. 
    \begin{itemize}
        \item Donner le nombre total de personnes interrogées.
        \item Donner le nombre d'individus qui parlent Anglais, Espagnol et Italien.
        \item Donner le nombre d'individus qui parlent Espagnol et Italien seulement.
        \item Donner le nombre d'individus qui parlent (Anglais et Italien) ou (Anglais et Espagnol)
    \end{itemize}
\end{minipage}
\tocless\subsection{}
Soient A et B deux ensembles finis et disjoints. On sait que Card(A$\cup$B)=23 et Card(A×B)=132. Déterminer Card(A) et Card(B) sachant que Card(A)<Card(B).
\tocless\section{Dénombrement}
\tocless\subsection{}
\begin{itemize}
    \item Combien y-a-t-il de podiums possibles?
    \item Combien y-a-t-il de podiums possibles où Émile est premier?
    \item Combien y-a-t-il de podiums possibles dont Émile fait partie?
    \item On souhaite récompenser les 3 premiers en leur offrant un prix identique à chacun. Combien y-a-t-il de distributions de récompenses possibles?
\end{itemize}
\tocless\subsection{}
Un cadenas possède un code à 3chiffres, chacun des chiffres pouvant être un chiffre de 1 à 9.
\begin{itemize}
    \item Combien y-a-t-il de codes possibles?
 \item Combien y-a-t-il de codes se terminant par un chiffre pair?
 \item Combien y-a-t-il de codes contenant au moins un chiffre 4?
 \item Combien y-a-t-il de codes contenant exactement un chiffre 4?
\end{itemize}
Dans cette question on souhaite que le code comporte obligatoirement trois chiffres distincts.
\begin{itemize}
     \item Combien y-a-t-il de codes possibles?
\item Combien y-a-t-il de codes se terminant par un chiffre impair?
 \item Combien y-a-t-il de codes comprenant le chiffre 
6
?
\end{itemize}
