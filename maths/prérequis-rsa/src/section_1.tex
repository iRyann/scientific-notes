\section{Arithmétique élémentaire}

\subsection{Divisibilité}

Soient $a,b \in \mathbb{Z}$ avec $b \neq 0$.
On dit que $b$ divise $a$, et l’on note $b \mid a$, s’il existe $q \in \mathbb{Z}$ tel que
\[
a = bq.
\]

\subsection{Plus grand commun diviseur}

Soient $a,b \in \mathbb{Z}$, non tous deux nuls.
Le \emph{plus grand commun diviseur} de $a$ et $b$, noté $\operatorname{pgcd}(a,b)$, est l’unique entier naturel vérifiant :
\begin{itemize}
  \item $\operatorname{pgcd}(a,b) \mid a$ et $\operatorname{pgcd}(a,b) \mid b$,
  \item tout diviseur commun de $a$ et $b$ divise $\operatorname{pgcd}(a,b)$.
\end{itemize}

\subsection{Théorème de Bézout}

\begin{theorem}[Bézout]
Soient $a,b \in \mathbb{Z}$.
Il existe des entiers $u,v \in \mathbb{Z}$ tels que
\[
au + bv = \operatorname{pgcd}(a,b).
\]
\end{theorem}

En particulier,
\[
\operatorname{pgcd}(a,b) = 1 \iff \exists u \in \mathbb{Z},\ au \equiv 1 \ (\mathrm{mod}\ b).
\]

\section{Congruences et anneaux quotient}

\subsection{Congruence modulo $n$}

Soit $n \geq 2$ un entier.
Deux entiers $a,b$ sont dits congrus modulo $n$ si
\[
n \mid (a-b),
\]
ce que l’on note
\[
a \equiv b \ (\mathrm{mod}\ n).
\]

Cette relation est une relation d’équivalence sur $\mathbb{Z}$.

\subsection{Anneau $\mathbb{Z}/n\mathbb{Z}$}

L’ensemble des classes de congruence modulo $n$ est noté
\[
\mathbb{Z}/n\mathbb{Z}.
\]

Il est muni de deux lois :
\[
\overline{a} + \overline{b} = \overline{a+b}, \qquad
\overline{a} \cdot \overline{b} = \overline{ab}.
\]

C’est un anneau commutatif unitaire.

\section{Inversibilité modulo $n$}

\subsection{Caractérisation}

\begin{theorem}
Soient $a \in \mathbb{Z}$ et $n \geq 2$.
L’élément $\overline{a} \in \mathbb{Z}/n\mathbb{Z}$ est inversible pour la multiplication
si et seulement si
\[
\operatorname{pgcd}(a,n) = 1.
\]
\end{theorem}

\subsection{Groupe multiplicatif}

On définit
\[
(\mathbb{Z}/n\mathbb{Z})^{\times}
=
\{\, \overline{a} \in \mathbb{Z}/n\mathbb{Z} \mid \operatorname{pgcd}(a,n)=1 \,\}.
\]

Cet ensemble est un groupe abélien pour la multiplication.

\section{Indicatrice d’Euler}

\subsection{Définition}

On appelle \emph{indicatrice d’Euler} de $n$ l’entier
\[
\varphi(n) = \#(\mathbb{Z}/n\mathbb{Z})^{\times}.
\]

Autrement dit, $\varphi(n)$ est le nombre d’entiers entre $1$ et $n$ premiers avec $n$.

\subsection{Cas particulier}

Si $p$ est premier,
\[
\varphi(p) = p-1.
\]

Si $p$ et $q$ sont deux nombres premiers distincts,
\[
\varphi(pq) = (p-1)(q-1).
\]

\section{Théorème d’Euler}

\begin{theorem}[Euler]
Soit $n \geq 2$ et $a \in \mathbb{Z}$ tel que $\operatorname{pgcd}(a,n)=1$.
Alors
\[
a^{\varphi(n)} \equiv 1 \ (\mathrm{mod}\ n).
\]
\end{theorem}

\section{Exponentiation modulaire}

Pour calculer $a^{k} \ (\mathrm{mod}\ n)$, on utilise l’exponentiation rapide,
fondée sur la décomposition binaire de $k$ et la réduction modulo $n$ à chaque étape.

La complexité est logarithmique en $k$.

\section{Principe mathématique du chiffrement RSA}

\subsection{Construction des clés}

On choisit deux nombres premiers distincts $p$ et $q$.

On définit
\[
n = pq,
\qquad
\varphi(n) = (p-1)(q-1).
\]

On choisit un entier $e$ tel que
\[
1 < e < \varphi(n)
\quad \text{et} \quad
\operatorname{pgcd}(e,\varphi(n))=1.
\]

Par le théorème de Bézout, $e$ est inversible modulo $\varphi(n)$.
On définit $d$ par
\[
ed \equiv 1 \ (\mathrm{mod}\ \varphi(n)).
\]

\subsection{Clés}

\begin{itemize}
  \item Clé publique : $(e,n)$
  \item Clé privée : $d$ (ou $(d,n)$)
\end{itemize}

\section{Chiffrement et déchiffrement}

Soit $m \in \mathbb{N}$ tel que $m < n$.

\subsection{Chiffrement}

\[
c \equiv m^{e} \ (\mathrm{mod}\ n).
\]

\subsection{Déchiffrement}

\[
m' \equiv c^{d} \ (\mathrm{mod}\ n).
\]

\subsection{Justification mathématique}

Comme $ed \equiv 1 \ (\mathrm{mod}\ \varphi(n))$, il existe $k \in \mathbb{Z}$ tel que
\[
ed = 1 + k\varphi(n).
\]

Si $\operatorname{pgcd}(m,n)=1$, alors par le théorème d’Euler :
\[
m^{ed} = m^{1+k\varphi(n)} = m(m^{\varphi(n)})^{k} \equiv m \ (\mathrm{mod}\ n).
\]

Dans le cas général, la validité repose sur le théorème des restes chinois.

\section{Méthode complète de calcul d’une clé RSA}

\begin{enumerate}
  \item Choisir deux nombres premiers $p \neq q$.
  \item Calculer $n = pq$.
  \item Calculer $\varphi(n) = (p-1)(q-1)$.
  \item Choisir $e$ tel que $\operatorname{pgcd}(e,\varphi(n))=1$.
  \item Calculer $d$ à l’aide de l’algorithme d’Euclide étendu.
  \item Former la clé publique $(e,n)$ et la clé privée $d$.
\end{enumerate}


