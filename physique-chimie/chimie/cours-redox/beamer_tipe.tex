\documentclass[compress]{beamer} %en animé
%\documentclass[trans]{beamer} %en superposé

%%%%%%%%%%%%%% LES PACKAGES
\input{les_packages_pour_beamer.tex}
%%%%%%%%%%%%%% THEME BEAMER
%votre nom court qui apparaîtra dans l'en tête
\newcommand{\initiales}{Ryan Bouchou}
\input{les_param_beamer_tipe}


%%%%%%%%%%%%%% MA PALETTE DE COULEURS
\input{la_palette.tex}
\definecolor{hellseahorse}{RGB}{204, 204, 255}
\definecolor{seahorse}{RGB}{204,180, 255}
\definecolor{darkseahorse}{RGB}{83, 74, 196}

%symboles pratiques
\newcommand{\flch}{\item[$\rightarrow$]}
\newcommand{\dc}{{\usebeamercolor[fg]{structure}$\hookrightarrow$}}
\newcommand{\ok}{\textcolor{vert}{\checkmark}}
\newcommand{\point}{{\usebeamercolor[fg]{structure}$\bullet\enskip$}}
\newcommand{\Point}{{\usebeamercolor[fg]{structure}$\bullet\enskip$}}


%styles
\newcommand{\couleur}[1]{{\usebeamercolor[fg]{structure}#1}}
\newcommand{\important}[1]{\couleur{\textbf{#1}}}
\newcommand{\remarque}[1]{\textit{\textrm{#1}}}

%pour le template
\newcommand{\lin}[1]{\mintinline{latex}{#1}}
%%%%%%%%%%%%%%%%%%%%%%%%%%%%%%%%%%%%%%%%%%%%%%%%%
\begin{document}
%

%%%%%%%%%%%%%%%%%%%%%%%%%%%%%%%%%%%%%%%%%%%%%%%%%
% A COMPLETER 
%%%%%%%%%%%%%%%%%%%%%%%%%%%%%%%%%%%%%%%%%%%%%%%%%

%entre crochet le titre court pour l'en tête puis le vrai titre
\title[Chimie Première]{
Chimie :\\
{\large Etude de la matière }
}

\author{
\large 
Présentation de \important{Ryan BOUCHOU}\\[0.2cm]
%\footnotesize
%travail réalisé avec \couleur{binôme}\\[0.4cm]
\vspace*{-1.2cm}
} 

%entre crochet la short date pour l'en-tête
\date[Novembre 2023]{}

\institute{}


%------------------------------------------------
\begin{frame}[plain]
\titlepage 
\addtocounter{framenumber}{-1} 
\end{frame}

%------------------------------------------------
\title[Cours de chimie: Première]{
Une page de titre avec logos :\\
{\large mais sans les affectations détaillées des auteurs }
}

%------------------------------------------------


\begin{frame}[plain]                           %|
\frametitle{Plan\\[0.1cm]}                     %|
\tableofcontents[                              %|
%currentsection,                                %|
hidesubsections,                          %|
%subsubsectionstyle=hide
]                       %|
\addtocounter{framenumber}{-1}                 %|
\end{frame}
\section{Notions générales}

\begin{frame}{Quantifier la matière\esp}
    Les quantités de matières sont exprimées en \textbf{moles}.
    \begin{center}
    \begin{minipage}{0.5\textwidth}
    \centering
        \fcolorbox{vertdEau}{white}{
            1 mole = $\underbrace{6,023.10^{23}}_{\mathcal{N}_{A}}$ entités
        }
    \end{minipage}
    \begin{minipage}{0.4\textwidth}
        \centering
         \includegraphics[scale=0.3]{src/img/mole.png}
    \end{minipage}
   \end{center}
    \pause

    La quantité de matière se définit comme suit:
    \begin{center}
        \fcolorbox{vertdEau}{white}{

        $ n= \frac{m}{M},~ où~
        \begin{cases}
            m & \text{masse en}~g\\
            M & \text{masse molaire en}~g.mol^{-1}\\
        \end{cases}
        $
        }
    \end{center}
    \remarque{Remarque: Peut importe l'état de l'échantillon.}
\end{frame}
\begin{frame}{Masse molaire\esp}
    \begin{block}{Définition}
        La masse molaire moléculaire est la somme des masses molaires atomiques.
    \end{block}
    \pause
    Si on considère une molécule de méthane, on a:
    \begin{align*}
        M(CH_{4}) 
        & = M(C)+4*M(H) \\
        & = 12,01+4*1,008\\
        & = 16,04 ~ g.mol^{-1}
    \end{align*}
    \remarque{On retrouve les masses molaires atomiques dans la classification périodique}
\end{frame}
\begin{frame}{Les autres relations\esp}
    
    \begin{table}
        \centering
        \begin{tabular}{c|c}
            Masse volumique & Densité\\
            \hline

        $ \rho= \frac{m}{V},~ \footnotesize{où~
        \begin{cases}
            m & \text{masse en}~Kg\\
            V & \text{volume en}~m^{3}\\
        \end{cases}}
        $
        & 

        $ d= \frac{\rho}{\rho_{eau}},~ 
        \footnotesize{où~\rho~\text{m. volumique en}  ~ Kg.m^{-3}}$
        \\
        \hline\hline
        Qte de matière & Concentration molaire\\
        \hline
        $ n= \frac{\rho*V}{M}
        $&$ [X]= \frac{n_{X}}{V},~ \tiny{où~
        \begin{cases}
            n_{X} & \text{qte de mat. de X en}~mol\\
            V & \text{volume de la solution en}~L\\
            [X] & \text{en}~mol.L^{-1}
        \end{cases}}
        $\\
        \hline\hline
        Relation entre c et $c_m$ & Concentration massique\\
        \hline
        $ c = \frac{c_m}{M}
        $&$ c_m= \frac{m}{V},~ \tiny{où~
        \begin{cases}
            m & \text{masse de X en}~g\\
            V & \text{volume de la solution en}~L\\
            c_m & \text{en}~g.L^{-1}
        \end{cases}}
        $
        \end{tabular}
    \end{table}
    \remarque{\footnotesize Remarque: 1 $mol.l^{-1}$ = 1 000 $mol.m^{-3}$}
\end{frame}
\section{Évolution d'un système chimique}
\begin{frame}{Caractéristiques d'une réaction}
    On s'intéresse ici à l'évolution des composants d'une réaction chimique.
    \pause
    \begin{block}{Définitions}
        \begin{enumerate}
            \item \textbf{Réactif}: On appelle réactif une espèce chimique présente à l'état initial, vouée à être consommée.
            \item \textbf{Produit}: Espèce chimique formée au cours de la réaction à partir des réactifs.
            \end{enumerate}
    \end{block}
    \pause
    Lors d'une transformation chimique, on quantifie l'évolution du système par \couleur{l'évolution des quantités de matières des espèces chimiques}.
    \pause
    \begin{itemize}
        \flch La quantité de matière des réactifs diminue.
        \flch La quantité de matière des produits augmente.
    \end{itemize}
\end{frame}
\begin{frame}{Loi fondamentale\esp}
    \begin{block}{Principe de Lavoisier}
        Au cours d'une réaction chimique, la masse se converse. De fait, la conservation de la matière implique:
        \begin{itemize}
            \flch conservation du nombre d'atomes de chaque élément chimique ;
            \flch conservation de la charge globale.
        \end{itemize}
    \end{block}
    En conséquence de quoi, la réaction est modélisée par une équation de conservation qui met en jeu les réactifs et les produits. 
\end{frame}
\subsection{Nombres stoechiométrique}
\begin{frame}{Coefficients stoechiométriques}
    Les \couleur{coefficients stoechiométriques} permettent d'équilibrer les espèces chimiques de la réaction. Ils indiquent donc leurs proportions.\\
    \begin{block}{Notation}
        De façon générale, une équation réactionnelle suivra la notation forme suivante:\\
        $$\alpha_1 A_1+...+\alpha_k A_k \rightleftarrows \alpha_{k+1} A_{k+1}+...+\alpha_n A_n$$
        où les $\alpha_i$ sont les c.s et $A_i$ les espèces chimiques.
    \end{block}
\end{frame}
\begin{frame}{Exemple}
    Considérons une transformation chimique:
    \only<1>{
    $$\underbrace{\textcolor{briqueRouge}{\alpha}N_2_{(g)} + \textcolor{briqueRouge}{\beta} H_2_{(g)}}_{\text{réactifs}}= \underbrace{\textcolor{briqueRouge}{\gamma} NH_3_{(g)}}_{\text{produit}}$$
    \textbf{\couleur{Déterminer les nombres stoechiométriques.}}
    }
    \only<2->{
    $$\underbrace{N_2_{(g)} + 3 H_2_{(g)}}_{\text{réactifs}}= \underbrace{2 NH_3_{(g)}}_{\text{produit}}$$
    }
    
\end{frame}
\begin{frame}{Méthode\esp}
    Pour équilibrer une équation chimique, on procède comme suit:
    \only<1>{
    \begin{itemize}
        \flch De tête, lorsque l'équation est simple
        \flch algébriquement, lorsque ça n'est pas évident.
    \end{itemize}
    }
    \only<2->{
    \begin{enumerate}
        \item On identifie dans l'énoncé les différentes espèces et leur rôle.
        \pause
        \item On pose l'équation chimique: $C_6H_{12}O_6 + O_2 = CO_2 + H_2O$
        \pause
        \item On porte une variable algébrique sur chaque réactif ou produit:
         $$\textbf{a}C_6H_{12}O_6 + \textbf{b}O_2 = \textbf{x}CO_2 + \textbf{y}H_2O$$
         \vspace*{-0.5cm}
        \pause
        \item On exprime le nombre d'atomes de nature différente:
        $$\left\lbrace \begin{array}{ll}
        C: & 6a = x\\
        H: & 12a = 2y\\
        O: & 6a + 2b = 2x + y
        \end{array}\right.
        $$
        \pause
        \item Dans le cas présent, on pose une valeur arbitraire pour une des variables, ce qui permettra de résoudre le système d'équations.
    \end{enumerate}
    }
\end{frame}
\begin{frame}{Méthode\esp}
    \remarque{Remarque: Si on obtient des coefficients fractionnaires, on détermine un dénominateur commun et on multiplie chacun des coefficients par celui-ci.}\\[0.5cm]
    Ici, en posant \textbf{a=1}, on obtient:
    $$C_6H_{12}O_6 + \textbf{6}O_2 = \textbf{6}CO_2 + \textbf{6}H_2O$$
\end{frame}
\begin{frame}{Exercice\esp}
    Équilibrer les équations suivantes:
    \begin{enumerate}
        \item $NH_3 + O_2 \rightarrow NO + H_2O$
        \item $CO + Fe_3O_4 \rightarrow CO_2 + Fe	$
        \item $Cu_2S + Cu_2O \rightarrow Cu + SO_2$
        \item $CH_4 + H_2O \rightarrow CO_2 + H_2	$
    \end{enumerate}
    \label{exo1}
\end{frame}
\begin{frame}{Nb. sto. algébriques}
    Les \textbf{\couleur{coefficients stoechiométriques algébriques}}, notés $\nu_i$, des différentes espèces chimiques $A_i$ intervenant dans la réaction sont :
    $$\nu_i = \begin{cases}
-\alpha_i & \text{si } A_i ~\text{est un réactif}\\
\alpha_i & \text{si } A_i ~\text{est un produit}\\
\end{cases}$$
Dans le cas précédent, on aurait:
\begin{center}
\begin{minipage}{0.4\textwidth}
    $\underbrace{N_2_{(g)} + 3 H_2_{(g)}}_{\text{réactifs}}= \underbrace{2 NH_3_{(g)}}_{\text{produit}}$
\end{minipage}
\begin{minipage}{0.3\textwidth} 
    \begin{itemize}
        \item $\nu(NH_3) = +2$
        \item $\nu(N_2) = −1$
        \item $\nu(H_2) = −3$
    \end{itemize}
\end{minipage}
\end{center}
\end{frame}
\subsection{Avancement}
\begin{frame}{Avancement\esp}
    Lorsque les réactifs sont tous épuisés, on dit que que \important{la réaction est totale}. Dans ce cas, les éléments sont en \important{conditions stoechiométriques}
    $$\forall i,j\in[1..n]^{2}, ~\frac{c_i}{\alpha_i}=\frac{c_j}{\alpha_j}~,~\text{où}\begin{cases}
        c_i & \text{concentration de}~A_i\\
        \alpha_i & \text{coeff. sto. de}~A_i
    \end{cases}$$
    \pause
    Dans le cas contraire, il est nécessaire d'introduire \important{$x$, l'avancement de la réaction},\remarque{(aussi noté $\xi$)}, qui permet de quantifier en mol la quantité de matière consommée par un réactif ayant un coef. sto. égal à 1.\\[0.5cm]\pause
    Ainsi, au début de la réaction : $x(0) = 0$, et il est relié de façon générale au nombre de moles $n_i(t)$ à l’instant $t$ de l’espèce chimique $A_i$ par :
    \vspace*{-0.8cm}
    \begin{equ}[!ht]
      \begin{equation}
        n_i(t) = n_i(0) + \nu_i*x(t)
      \end{equation}
    \caption{}
    \label{eq1}
    \end{equ}
    
\end{frame}
\begin{frame}{Exemple\esp}
    Considérons la réaction: ${N_2}_{(g)} + 3 {H_2}_{(g)}= 2{NH_3}_{(g)}$ en supposant qu'à l'état initial, on a une quantité $n_0$ d'azote et de dihydrogène et qu'aucun des produit n'est présent.\pause
    
    \begin{table}
        \centering
        \begin{tabular}{|c|c|c|c|}
            \hline
             & ${N_2}_{(g)}$ & ${H_2}_{(g)}$ &${NH_3}_{(g)}$\\
             \hline\hline
             $t=0$& $n_0$ & $n_0$ & 0\\
             \hline
             $t>0$& $n_0-x(t)$ & $n_0-3x(t)$ & $2x(t)$\\
             \hline
        \end{tabular}
        \caption{Tableau d'avancement}
        \label{tab:my_label}
    \end{table}
    \pause
    De fait, si on s'intéresse à $n_{N_2}$ au cours du temps, on a bien:
    \begin{align}
n_{N_2}(t) 
& = n_{N_2}(0) + \nu_{N_2}*x(t) \label{q:1:1}\\
& = n_{N_2}(0) + (-1)*x(t)
\end{align}
\footnotesize{\remarque{Remarque: Dans \eqref{q:1:1}, on utilise la formule générale \eqref{eq1}}}
\end{frame}
\begin{frame}{Avancement maximal\esp}
    \begin{block}{Caractérisation}
        Valeur maximale de $x$, que l’on peut noter $x_{max}$. On la calcule en imposant que les quantités de matière de tous les réactifs soient positives ou nulles.
    \end{block}
    \pause \textbf{Exemple}:
    Considérons la réaction {\small
    $C_3H_8_{(g)} + 5O_2_{(g)} = 3 CO_2_{(g)} + 4 H_2O_{(l)}$}
    et notons {\small$n_1, n_2$} les quantités de matières de respectivement {\small$C_3H_8$} et {\small$O_2$}. \pause Par définition,
    $$n_1 - x \geq 0 ~et~ n_2 - 5x \geq 0$$
\pause donc
$$x \leq n_1 ~et~ x\leq \frac{n_2}{5}$$
\pause Dans ce cas on peut poser $x_{max}$ = min\{$n_1$,$\frac{n_2}{5}$\}; et donc, 
$$\forall t,~0\leq x(t)\leqx_{max}$$
\end{frame}
\begin{frame}{Avancement final\esp}
    Lorsque \textbf{la transformation chimique est terminée}, les quantités de matière n’évoluent plus et $x$ prend sa valeur finale $x_f$ : c’est l’\important{avancement final}.\\
    \pause
    On distingue alors deux types de réactions:
    \begin{enumerate}
        \item Les \important{réactions totales ($x_f=x_{max}$)}\\
        Dans ce cas, la réaction est se poursuit jusqu’à la disparition d’un
des réactifs, appelé \textit{réactif limitant}. La transformation chimique se note à l’aide d’une flèche $\rightarrow$ pour indiquer qu’elle est totale.
    \pause
    \item Les \important{réactions limitées ($x_f<x_{max}$)}\\
    Dans ce cas, la réaction aboutit à un état final où coexistent tous
les réactifs et les produits. On dit qu’il y a équilibre chimique. Une réaction limitée est la superposition de deux réactions totales en sens
inverse. Ce faisant, des produits sont consommés au fur et à mesure qu'ils sont créés pour reformer des réactifs.
    \end{enumerate}
\end{frame}
\subsection{Bilan}
\begin{frame}{}
    \begin{itemize}
        \item Les quantités initiales ne dépendent pas des coefficients stoechiométriques.
        \item Le tableau d'avancement permet de suivre l'évolution système, mais surtout de connaître l'état final.
    \end{itemize}
    
    \begin{table}
        \centering
        \begin{tabular}{|c|c||c||c|}
            \hline
             & $A_1$ \remarque{(\footnotesize réactif)} & ... & $A_n$ \remarque{(\footnotesize produit)} \\
             \hline
             $t=0$ & ${n_1}_{initial}$& ... & ${n_n}_{initial}$ \\
             \hline
             $t_{final}$ & ${n_1}_{initial}-\alpha_1x_{max}$ & ... & ${n_n}_{initial} + \alpha_nx_{max}$\\
             \hline
        \end{tabular}
        \caption{Tableau d'avancement}
        \label{tab:my_label}
    \end{table}
    \begin{itemize}
        \item Si la réaction est totale, et que tous les réactifs sont épuisés, les espèces chimiques sont présentes en conditions stoechiométriques. On parle de mélange stoechiométriques.
    \end{itemize}
\end{frame}
\begin{frame}{Exercice\esp}
    \begin{figure}
        \centering
        \includegraphics[width=0.5\linewidth]{src/img/image1.png}
        \caption{Exercice - LLS}
        \label{lls1}
    \end{figure}
\end{frame}

\section{Équilibres d'oxydoréduction}
\begin{frame}{Couple Redox}
    \begin{block}{Définitions}
        \begin{itemize}
            \flch \important{Oxydant}: Espèce chimique susceptible de capter un ou plusieurs électrons.
            \flch \important{Réducteur}: Espèce chimique susceptible de céder un ou plusieurs électrons.
        \end{itemize}
    \end{block}
    \pause
    Tout naturellement, un oxydant ayant capté des électrons, est à son tour un réducteur et réciproquement. \pause De fait, il est possible d'établir des couples Red/Ox liés par une demi-équation électronique:
    $$Ox + ne^{-} + xH^{+} = Red + y H_2O$$
    Ou dans les cas les plus simples: $Ox + ne^{-} = Red$
\end{frame}
\begin{frame}{Réaction d'oxydoréduction}
    Une réaction d'oxydoréduction est un transfert d'électrons entre deux espèces chimiques.\\
    Un oxydant est \textit{réduit} lorsqu'il capte des électrons; et un réducteur est \textit{oxydé} lorsqu'il en cède.
\end{frame}
\subsection{Demi-équation}
\begin{frame}{Méthode - En milieu acide\esp}
    \begin{enumerate}
        \item Écrire la demi-équation $Ox + ne^{-} = Red$\pause
        \item Ajuster les nombres stoechiométrique pour assurer la conservation des éléments autres que $O$ et $H$\pause
        \item Assurer la conservation de l'oxygène en ajoutant des molécules d'eau\pause
        \item Assurer la conservation de l'hydrogène à l'aide de protons (ions $H^{+})$\pause
        \item Assurer la conservation de la charge avec des électrons
    \end{enumerate}
\end{frame}
\begin{frame}{Exemple - En milieu acide\esp}
    Considérons le couple $Cr_2O_{7}^{2-}/Cr^{3+}$\pause
    \begin{enumerate}
        \item Conservation du chrome: \hfill $Cr_2O_{7}^{2-} +ne^{-} = \boldsymbol{2}Cr^{3+}$\pause
        \item Conservation de l'oxygène: \hfill $Cr_2O_{7}^{2-} +ne^{-} = 2Cr^{3+} + \boldsymbol{7H_2O}$\pause
        \item Conservation de H: \hfill $Cr_2O_{7}^{2-} +ne^{-} + \boldsymbol{14H^{+}}= 2Cr^{3+} + H_2O$\pause
        \item Conservation de la charge: \hfill {\small$Cr_2O_{7}^{2-} +\boldsymbol{6}e^{-} + 14H^{+}= 2Cr^{3+} + H_2O$}
    \end{enumerate}
\end{frame}

\begin{frame}{Méthode - En milieu basique\esp}
    \begin{enumerate}
        \item Établir la demi-équation en milieu acide\pause
        \item Ajouter aux deux membres de la demi-éq. autant d'ions $HO^{-}$ qu'il y a de protons\pause
        \item Remplacer les $H^{+}+HO^{-}$ par des molécules d'eau\pause
        \item Simplifier au besoin
    \end{enumerate}
\end{frame}
\begin{frame}{Établir l'équation de réaction\esp}
    \begin{enumerate}
        \item Établir les couples mis en jeu et écrire les deux demis-équations
        \item Combiner linéairement les deux équations afin qu'il ne reste plus d'électrons
    \end{enumerate}
\end{frame}
\begin{frame}{Exercice\esp}
    Établir les demis-équations électroniques des couples suivant en milieu acide:
    \begin{enumerate}
        \item $Fe^{3+}/Fe^{2+}$
        \item $IO^{-}_3/I^{-}$
        \item $MnO^{-}_4/Mn^{2+}$
        \item $Fe(CN)^{3+}_6/Fe^{2+}$
        \item $Hg_2Cl_2/Hg_s$
    \end{enumerate}
    Établir les demis-équations électroniques des couples suivant en milieu basique:
    \begin{enumerate}
        \item $MnO^{-}_4/Mn^{2+}$
        \item $Al(OH)^{-}_4/Al$
    \end{enumerate}
\end{frame}
\begin{frame}{Exercice\esp}
    \begin{figure}
        \centering
        \includegraphics[width=0.5\linewidth]{src//img/image3.png}
        \caption{Exo - LLS}
        \label{lls2}
    \end{figure}
\end{frame}
\begin{frame}{Exercice\esp}
    
\end{frame}
\begin{frame}{Correction \ref{exo1}}
    \begin{enumerate}
        \item $4NH_3 + 5O_2 \rightarrow 4NO + 6H_2O$
        \item $4CO + Fe_3O_4 \rightarrow 4CO_2 + 3Fe	$
        \item $Cu_2S + 2Cu_2O \rightarrow 6Cu + SO_2$
        \item $CH_4 + 2H_2O \rightarrow CO_2 + 4H_2	$
    \end{enumerate}
\end{frame}
\begin{frame}{Exercice \ref{lls1}\esp}
    \begin{figure}
        \centering
        \includegraphics[width=0.7\linewidth]{image.png}
        \caption{Cor. LLS}
        
    \end{figure}
\end{frame}

\end{document}
