\section{Équilibres d'oxydoréduction}
\begin{frame}{Couple Redox}
    \begin{block}{Définitions}
        \begin{itemize}
            \flch \important{Oxydant}: Espèce chimique susceptible de capter un ou plusieurs électrons.
            \flch \important{Réducteur}: Espèce chimique susceptible de céder un ou plusieurs électrons.
        \end{itemize}
    \end{block}
    \pause
    Tout naturellement, un oxydant ayant capté des électrons, est à son tour un réducteur et réciproquement. \pause De fait, il est possible d'établir des couples Red/Ox liés par une demi-équation électronique:
    $$Ox + ne^{-} + xH^{+} = Red + y H_2O$$
    Ou dans les cas les plus simples: $Ox + ne^{-} = Red$
\end{frame}
\begin{frame}{Réaction d'oxydoréduction}
    Une réaction d'oxydoréduction est un transfert d'électrons entre deux espèces chimiques.\\
    Un oxydant est \textit{réduit} lorsqu'il capte des électrons; et un réducteur est \textit{oxydé} lorsqu'il en cède.
\end{frame}
\subsection{Demi-équation}
\begin{frame}{Méthode - En milieu acide\esp}
    \begin{enumerate}
        \item Écrire la demi-équation $Ox + ne^{-} = Red$\pause
        \item Ajuster les nombres stoechiométrique pour assurer la conservation des éléments autres que $O$ et $H$\pause
        \item Assurer la conservation de l'oxygène en ajoutant des molécules d'eau\pause
        \item Assurer la conservation de l'hydrogène à l'aide de protons (ions $H^{+})$\pause
        \item Assurer la conservation de la charge avec des électrons
    \end{enumerate}
\end{frame}
\begin{frame}{Exemple - En milieu acide\esp}
    Considérons le couple $Cr_2O_{7}^{2-}/Cr^{3+}$\pause
    \begin{enumerate}
        \item Conservation du chrome: \hfill $Cr_2O_{7}^{2-} +ne^{-} = \boldsymbol{2}Cr^{3+}$\pause
        \item Conservation de l'oxygène: \hfill $Cr_2O_{7}^{2-} +ne^{-} = 2Cr^{3+} + \boldsymbol{7H_2O}$\pause
        \item Conservation de H: \hfill $Cr_2O_{7}^{2-} +ne^{-} + \boldsymbol{14H^{+}}= 2Cr^{3+} + H_2O$\pause
        \item Conservation de la charge: \hfill {\small$Cr_2O_{7}^{2-} +\boldsymbol{6}e^{-} + 14H^{+}= 2Cr^{3+} + H_2O$}
    \end{enumerate}
\end{frame}

\begin{frame}{Méthode - En milieu basique\esp}
    \begin{enumerate}
        \item Établir la demi-équation en milieu acide\pause
        \item Ajouter aux deux membres de la demi-éq. autant d'ions $HO^{-}$ qu'il y a de protons\pause
        \item Remplacer les $H^{+}+HO^{-}$ par des molécules d'eau\pause
        \item Simplifier au besoin
    \end{enumerate}
\end{frame}
\begin{frame}{Établir l'équation de réaction\esp}
    \begin{enumerate}
        \item Établir les couples mis en jeu et écrire les deux demis-équations
        \item Combiner linéairement les deux équations afin qu'il ne reste plus d'électrons
    \end{enumerate}
\end{frame}
\begin{frame}{Exercice\esp}
    Établir les demis-équations électroniques des couples suivant en milieu acide:
    \begin{enumerate}
        \item $Fe^{3+}/Fe^{2+}$
        \item $IO^{-}_3/I^{-}$
        \item $MnO^{-}_4/Mn^{2+}$
        \item $Fe(CN)^{3+}_6/Fe^{2+}$
        \item $Hg_2Cl_2/Hg_s$
    \end{enumerate}
    Établir les demis-équations électroniques des couples suivant en milieu basique:
    \begin{enumerate}
        \item $MnO^{-}_4/Mn^{2+}$
        \item $Al(OH)^{-}_4/Al$
    \end{enumerate}
\end{frame}
\begin{frame}{Exercice\esp}
    \begin{figure}
        \centering
        \includegraphics[width=0.5\linewidth]{src//img/image3.png}
        \caption{Exo - LLS}
        \label{lls2}
    \end{figure}
\end{frame}
\begin{frame}{Exercice\esp}
    
\end{frame}