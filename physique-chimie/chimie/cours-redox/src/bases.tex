\section{Notions générales}

\begin{frame}{Quantifier la matière\esp}
    Les quantités de matières sont exprimées en \textbf{moles}.
    \begin{center}
    \begin{minipage}{0.5\textwidth}
    \centering
        \fcolorbox{vertdEau}{white}{
            1 mole = $\underbrace{6,023.10^{23}}_{\mathcal{N}_{A}}$ entités
        }
    \end{minipage}
    \begin{minipage}{0.4\textwidth}
        \centering
         \includegraphics[scale=0.3]{src/img/mole.png}
    \end{minipage}
   \end{center}
    \pause

    La quantité de matière se définit comme suit:
    \begin{center}
        \fcolorbox{vertdEau}{white}{

        $ n= \frac{m}{M},~ où~
        \begin{cases}
            m & \text{masse en}~g\\
            M & \text{masse molaire en}~g.mol^{-1}\\
        \end{cases}
        $
        }
    \end{center}
    \remarque{Remarque: Peut importe l'état de l'échantillon.}
\end{frame}
\begin{frame}{Masse molaire\esp}
    \begin{block}{Définition}
        La masse molaire moléculaire est la somme des masses molaires atomiques.
    \end{block}
    \pause
    Si on considère une molécule de méthane, on a:
    \begin{align*}
        M(CH_{4}) 
        & = M(C)+4*M(H) \\
        & = 12,01+4*1,008\\
        & = 16,04 ~ g.mol^{-1}
    \end{align*}
    \remarque{On retrouve les masses molaires atomiques dans la classification périodique}
\end{frame}
\begin{frame}{Les autres relations\esp}
    
    \begin{table}
        \centering
        \begin{tabular}{c|c}
            Masse volumique & Densité\\
            \hline

        $ \rho= \frac{m}{V},~ \footnotesize{où~
        \begin{cases}
            m & \text{masse en}~Kg\\
            V & \text{volume en}~m^{3}\\
        \end{cases}}
        $
        & 

        $ d= \frac{\rho}{\rho_{eau}},~ 
        \footnotesize{où~\rho~\text{m. volumique en}  ~ Kg.m^{-3}}$
        \\
        \hline\hline
        Qte de matière & Concentration molaire\\
        \hline
        $ n= \frac{\rho*V}{M}
        $&$ [X]= \frac{n_{X}}{V},~ \tiny{où~
        \begin{cases}
            n_{X} & \text{qte de mat. de X en}~mol\\
            V & \text{volume de la solution en}~L\\
            [X] & \text{en}~mol.L^{-1}
        \end{cases}}
        $\\
        \hline\hline
        Relation entre c et $c_m$ & Concentration massique\\
        \hline
        $ c = \frac{c_m}{M}
        $&$ c_m= \frac{m}{V},~ \tiny{où~
        \begin{cases}
            m & \text{masse de X en}~g\\
            V & \text{volume de la solution en}~L\\
            c_m & \text{en}~g.L^{-1}
        \end{cases}}
        $
        \end{tabular}
    \end{table}
    \remarque{\footnotesize Remarque: 1 $mol.l^{-1}$ = 1 000 $mol.m^{-3}$}
\end{frame}